%! Author = m.pielaszkiewicz
%! Date = 04.04.2023


\subsection{Aktywacja dowódcy}
\begin{wrapfigure}{l}{0.1\textwidth}
  \includegraphics[width=2cm, height=2cm]{leader_activation.png}
\end{wrapfigure}
Akcja aktywacji dowódcy umożliwia wykonanie jednej lub wielu następujących po sobie akcji marszu kwalifikującymi się do tego jednostkami w trakcie trwania tej samej fazy akcji. Następujące akcje \textbf{nie mogą być wykonywane} podczas aktywacji dowódcy: \textbf{szturmy, niszczenie stacji kolejowych oraz (w grach zaawansowanych) ruch jednostek taborem kolejowym}. Jedynie jeden dowódca regionalny, korpusu bądź dywizji może zostać wskazany do aktywacji. \textbf{Dowódca może zostać aktywowany jeśli choć jedna przynależna mu jednostka jest w zasięgu dowodzenia i jednocześnie posiada poziom zmęczenia 3 lub mniej}.\par
Procedura aktywacji dowódcy:
\begin{itemize}
  \item[1]
        Gracz z inicjatywą wyznacza jednego kwalifikującego się do aktywacji dowódcę.
  \item[2]
        Gracz musi wskazać jedną bądź więcej jednostek znajdujących się pod rozkazami wskazanego dowódcy oraz będących w jego zasięgu dowodzenia.    Wszystkie z wskazanych jednostek muszą posiadać poziom zmęczenia 3 i mniej. Gracz umieszcza żeton pomocniczy \"leader activation\" na każdej z tych jednostek. Gracz nie jest zobowiązany do wybrania każdej kwalifikującej się do tego jednostki, lecz może tak uczynić. W gestii gracza pozostaje wybór ilości aktywowanych przez danego dowódcę jednostek. Gracz musi wybrać w ten sposób zawsze minimum jedną jednostkę. W każdej grze będą jednostki które nie przynależą do żadnej struktury dowodzenia(np. jednostki kawalerii Unii w SIV, OTR, SJW, HCR, SLB oraz AGA nie przynależą do struktur dowodzenia gdyż nie ma w tych tytułach dowódców kawalerii po stronie Unii). \textbf{Jednostki bez przynależności do struktury dowodzenia nie mogą być wybrane podczas aktywacji dowódcy.} Mogą one zostać aktywowane wyłącznie indywidualnie.\par
        Wyjątki: \textbf{jednostki artylerii mogą zostać wskazane} podczas aktywacji dowódcy nie będącego dowódcą kawalerii nawet \textbf{jeśli nie przynależą do struktury dowodzenia aktywowanego dowódcy}. Dotyczy wszystkich tytułów serii.\par
        Jeśli w grze korzysta się z zasady opcjonalnej budowania umocnień w latach 1963 oraz 1964. Gracz musi wskazać dokładnie które z jednostek będą budowały umocnienia polowe, a które wykonają ruch przed wykonaniem rzutu na określenie wartości punktów ruchu dowódcy. Gracz może dowolnie przeznaczyć jednostki do ww. akcji, lecz istotnym jest aby deklaracja została sformułowana przed wykonaniem ww. rzutu w pt.3 procedury aktywacji dowódcy.\par
        \textcolor{violet}{\textbf{Opcjonalnie: }Jednostka na poziomie zmęczenia 4 może zostać wybrana do wykonania przemarszu (nie do budowania umocnień polowych) - jeżeli jej żeton siły jest na stronie zorganizowanej. Po wykonaniu przemarszu jej żeton siły jest obracany na stronę zdezorganizowaną. W przypadku takim gracz nie odnosi się do tabeli marszu wydłużonego oraz nie może wykonać marszu forsownego. - jest to analogia do zasady opcjonalnej w akcji marszu indywidualnej jednostki.}\par
        \textcolor{brown}{\textbf{Zasada specjalna w OTR: }\par
        W grze OTR jeśli gracz Unii \textbf{wygrał inicjatywę} to aktywując dowódcę musi określić ile jednostek może zostać aktywowanych przez dowódcę na podstawie rzutu kością na inicjatywę jaki wykonał:
        \begin{itemize}
          \item[wynik rzutu : 1-4] Gracz Unii może aktywować 1 jednostkę.
          \item[wynik rzutu : 5] Gracz Unii może aktywować 2 jednostki.
          \item[wynik rzutu : 6] Gracz Unii może aktywować dowolną liczbę jednostek.
        \end{itemize}
        \textbf{Uwaga:} Jeśli inicjatywa została zdobyta poprzez \textbf{spasowanie Konfederacji bądź bez rzutu na inicjatywę} (np. gracz Konfederacji nie posiadał jednostek kwalifikujących się do aktywacji) wówczas gracz Unii \textbf{może aktywować dowolną ilość} jednostek niezależnie od wyniku rzutu kością na inicjatywę lub jego braku. 
  \item[3] Ilość punktów ruchu dowódcy / MA określa się następująco:
        \begin{itemize}
          \item \textbf{UNIA - dowódcy piechoty} - Wynik rzutu \textbf{jedną kostką plus 1} gracza Unii determinuje ilość punktów ruchu dowódcy.
                \textit{\textbf{Wyjątek:} W SIV, OTA oraz AGA Wynik rzutu jedną kostką gracza Unii determinuje ilość punktów ruchu dowódcy. Wynik nie jest zwiększony o 1. Minimalna ilość punktów ruchu to 2 nawet jeśli wyrzucona wartość to 1}
          \item \textbf{UNIA - dowódcy dywizji kawalerii} - Wynik rzutu \textbf{dwiema kośćmi plus 1} gracza Unii determinuje ilość punktów ruchu jednostki.
          \item \textbf{UNIA - dowódcy korpusów kawalerii} - Wynik rzutu \textbf{dwiema kośćmi plus 2} gracza Unii determinuje ilość punktów ruchu jednostki.
          \item \textbf{KONFEDERACJA - dowódcy piechoty} - Wynik rzutu \textbf{jedną kostką plus 2} gracza Konfederacji determinuje ilość punktów ruchu dowódcy.
                \textit{\textbf{Wyjątek:} W OTA oraz AGA Wynik rzutu modyfikowany jest plus 1}
          \item \textbf{KONFEDERACJA - dowódcy dywizji kawalerii} - Wynik rzutu \textbf{dwiema kośćmi plus 2} gracza Konfederacji determinuje ilość punktów ruchu jednostki.
          \item \textbf{KONFEDERACJA - dowódcy korpusów kawalerii} - Wynik rzutu \textbf{dwiema kośćmi plus 3} gracza Konfederacji determinuje ilość punktów ruchu jednostki.
        \end{itemize}
  \item[4] CDN
\end{itemize}