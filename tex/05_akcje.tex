%! Author = m.pielaszkiewicz
%! Date = 04.04.2023



\section{Akcje}
Gracz posiadający inicjatywę \textbf{musi} wykonać \textbf{jedną} akcję wyznaczając do niej jednostkę lub dowódcę. W grze występuje pięć rodzajów akcji:
\begin{itemize}
	\item \textbf{Marsz}
	\item \textbf{Aktywacja dowódcy}
	\item \textbf{Szturm}
	\item \textbf{Niszczenie stacji kolejowych (nie w SIV \& OTR)}
	\item \textbf{Budowanie umocnień polowych (po 1863r. włącznie)}
\end{itemize}
\subsection{Marsz}
\begin{wrapfigure}{l}{0.1\textwidth}
    \includegraphics[width=2cm, height=2cm]{ma_limit.png}
\end{wrapfigure}
Wyłącznie jedna jednostka (nie dowódca) może zostać wybrany do przeprowadzenia akcji marszowej. Jednostka może wykonać przemarsz wyłącznie jeśli jej poziom
zmęczenia/fatigue jest mniejszy lub równy 3. Przemarszu nie można wykonać jeśli poziom zmęczenia wynosi 4. Jednostka może być zarówno zorganizowana jak i zdezorganizowana. Jednostka może wykonać więcej niż jeden przemarsz w trakcie trwania tury, lecz nigdy więcej niż raz w trakcie trwania Fazy Akcji. Procedura przemarszu:
\begin{itemize}
	\item[1] Gracz z inicjatywą wyznacza jedną kwalifikującą się do przemarszu jednostkę na mapie. Ilość punktów ruchu jednostki / MA określa się następująco:
	\begin{itemize}
		\item \textbf{UNIA - piechota lub artyleria} - Wynik rzutu \textbf{jedną kostką} gracza Unii determinuje ilość punktów ruchu jednostki. 
		\item \textbf{KONFEDERACJA - piechota lub artyleria} - wynik rzutu \textbf{jedną kostką plus 1} gracza Konfederacji determinuje ilość punktów ruchu jednostki. \textit{\textbf{Wyjątek:} W OTR oraz AGA Wynik rzutu jedną kostką gracza Konfederacji determinuje ilość punktów ruchu jednostki. Wynik nie jest zwiększony o 1. Minimalna ilość punktów ruchu to 2 nawet jeśli wyrzucona wartość to 1}
		\item \textbf{UNIA - kawaleria} - Wynik rzutu \textbf{dwiema kośćmi} gracza Unii determinuje ilość punktów ruchu jednostki. 
		\item \textbf{KONFEDERACJA - kawaleria} - wynik rzutu \textbf{dwiema kośćmi plus 1} gracza Konfederacji determinuje ilość punktów ruchu jednostki.
	\end{itemize}
	\item[2] Poziom zmęczenia/fatigue wybranej jednostki zwiększony jest o 1. Jednostka otrzymuje zaktualizowany znacznik zmęczenia.
	Jeśli jednostka jest w stanie wypoczętym (żeton jednostki na stronie wypoczętej) a wartość zmęczenia osiągnęła wartość 3 lub 4, lub jednostka jest wyczerpana/exhausted (żeton jednostki na stronie wyczerpanej) a wartość zmęczenia osiągnęła wartość 2, 3 lub 4 wówczas jest to przemarsz wydłużony/extended march. Gracz musi odnieść się do tabeli przemarszu wydłużonego \textcolor{violet}{(patrz reguła opcjonalna poniżej)}. Przed przejściem do pt. 3, kawaleria przeciwnika w ZOC jednostki marszowej może wykonać wycofanie kawalerii/Cav. retreat [7.7]
	\item[3] Jednostka może powiększyć limit dostępnych punktów ruchu przeprowadzając przemarsz forsowny/force march. Punkty ruchu otrzymane dzięki wykorzystaniu marszu forsownego kumulują się z punktami uzyskanymi w pt.1 Jeśli gracz nie decyduje się na wykonanie przemarszu forsownego wówczas ilość dostępnych punktów ruchu jest wartością określoną w pt.1
	\item[4] Gracz umieszcza żeton dostępnych punktów ruchu na odpowiednim polu odpowiadającym wartości punktów ruchu jednostki.
	\item[5] Jednostka wykonuje przemarsz. W miarę wydawania punktów ruchu żeton dostępnych punktów ruchy należy odpowiednio aktualizować. Przemarsz dobiega końca kiedy jednostka wyczerpie limit dostępnych punktów ruchu bądź gracz z inicjatywą zakończy jej ruch przed wykorzystaniem dostępnego limitu punktów ruchu. Jeśli ilość punktów ruchu spadła do 0 na skutek ataku, akcja nie zostaje zakończona do czasu rozstrzygnięcia ataku.
\end{itemize}
\textcolor{violet}{\textbf{Opcjonalnie: }Jednostka na poziomie zmęczenia 4 może zostać wybrana do wykonania przemarszu w pt.1 jeżeli jej żeton siły jest na stronie zorganizowanej. Po wykonaniu przemarszu jej żeton siły jest obracany na stronę zdezorganizowaną. W przypadku takim gracz nie odnosi się do tabeli marszu wydłużonego oraz nie może wykonać marszu forsownego.}
\bigbreak
\textbf{MARSZ WYDŁUŻONY / EXTENDED MARCH} - tba
\bigbreak
\textbf{MARSZ FORSOWNY / FORCE MARCH} - tba