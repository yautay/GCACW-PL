%! Author = m.pielaszkiewicz
%! Date = 04.04.2023
\section{Wstęp}

Instrukcja stanowi zbiór zasad dla wszystkich gier z Cyklu \textbf{GCACW}. Gry opublikowane w ramach systemu do czasu powstania instrukcji wraz z oznaczeniami:
\begin{itemize}
    \item Stonewall Jackson’s Way (\textbf{SJW})
    \item Here Come The Rebels! (\textbf{HCR})
    \item Roads to Gettysburg (\textbf{RTG})
    \item Stonewall in the Valley (\textbf{SIV})
    \item Stonewall’s Last Battle (\textbf{SLB})
    \item On to Richmond! (\textbf{OTR})
    \item Grant Takes Command (\textbf{GTC})
    \item Battle Above the Clouds (\textbf{BAC})
    \item All Green Alike (\textbf{AGA})
    \item Stonewall Jackson’s Way (\textbf{SJW}) [reedycja]
    \item Atlanta Is Ours (\textbf{AIO})
    \item Here Come the Rebels\! (\textbf{HCR}) [reedycja]
    \item Roads to Gettysburg (\textbf{RTG}) [reedycja]
    \item Rebels in the White House (\textbf{RWH}) [reedycja]
    \item \textcolor{blue}{Hood Strikes North (\textbf{HSN})}

\end{itemize}
Dwa moduły zostały opublikowane w magazynie \textit{Skirmisher \#2}
\begin{itemize}
    \item Burnside Takes Command (\textbf{BTC})
    \item Rebels in the White House (\textbf{RWH})
\end{itemize}
Pierwsze sześć gier z cyklu opublikowane zostały wraz z indywidualnymi zasadami i zawierały pewne różnice. Zasady systemowe (dalej \textbf{ZS}) zostały wprowadzone w \textbf{Standard Series Rules Upgrade Kit} wraz z pierwszym numerem \textbf{The Skirmisher}. Publikacja zawierała ujednolicone reguły dla wszystkich obecnych gier w ramach systemu GCACW. Wraz z powstaniem ZS każda kolejna gra zawiera ZS oraz oddzielny zestaw unikalnych zasad wymaganych w ramach danej kampanii. Pierwszy tytuł wydany wraz z ZS to \textbf{GTC}. Od pierwszej edycji ZS, zostały opublikowane jedynie małe poprawki do istniejących zasad. Obecna edycja ZS nosi oznaczenie \textbf{1.4} Wszelkie zmiany wprowadzone w ramach tej wersji zostały oznaczone kolorem \textcolor{blue}{niebieskim} w celu łatwiejszej identyfikacji.\par
Niniejsze SZ jako najbardziej aktualne powinny być wykorzystane do aktualizacji zasad w sekcjach od 2.0 do 12.0 dla pierwszych sześciu tytułów z serii. Należy zwrócić uwagę na fakt występowania zasad szczegółowych dla poszczególnych kampanii które powinny być stosowane. Zestaw opublikowany w \textbf{The Skirmisher \#1} zawierał 130 żetonów potrzebnych do konwersji pierwszych pięciu gier z systemu do ZS. \textit{Żetony te nie powinny być wykorzystywane w reedycji \textbf{SJW, HCR, RTG} gdyż tytuły te zawierają już poprawiony zestaw żetonów.}\par
Mimo że ZS ubogacają realizm oraz wrażenia z rozgrywki zwłaszcza dla wcześniejszych tytułów, ich implementacja podnosi jednak nieznacznie złożoność rozgrywki. Przed rozpoczęciem rozgrywki gracze powinni uzgodnić według których zasad chcą rozegrać grę. Gdyż mimo faktu że ZS będąc zasadami \textit{oficjalnymi}, dopuszcza się wolę rozegrania dowolnej kampanii wg. własnego uznania w kontekście wykorzystanych reguł. Zaleca się aby ZS jeśli są wykorzystane w grze były wykorzystane w całości, a nie selektywnie (nie dotyczy zasad \textit{opcjonalnych}).