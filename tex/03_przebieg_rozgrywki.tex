%! Author = m.pielaszkiewicz
%! Date = 04.04.2023
\section{Przebieg rozgrywki}
Każda tura gry przebiega wg. następującej sekwencji:
\begin{itemize}
  \item[1] \textbf{Faza wydarzeń losowych}\par
        Faza jest rozgrywana wyłącznie jeśli scenariusz przewiduje taką okoliczność (nie dotyczy scenariuszy zaawansowanych - gdzie faza ta nie istnieje)
        Na podstawie rzutu dwoma kośćmi gracze odczytują wydarzenie losowe z tabeli wydarzeń losowych
  \item[2] \textbf{Faza przemieszczania dowódców}\par
        Dowódcy mogą zostać przeniesieni pomiędzy jednostkami podległymi
  \item[3] \textbf{Faza Cyklu Akcji [4.0]}\par
        \textbf{Faza akcji [4.1]} \par
        \begin{itemize}
          \item[A] \textbf{Segment inicjatywy [4.2]}\par
                Gracze rzucają kośćmi. Gracz który uzyska większy wynik wygrywa inicjatywę i musi podjąć akcje lub spasować.
                Jeśli jest remis, wówczas stosuje się specjalne zasady.
                \begin{itemize}
                  \item \textbf{BAC | AIO} ---
                        Unia lub Konfederacja - może zdobyć inicjatywę, w zależności od konkretnie wyrzuconej liczby.
                        W \textbf{BAC} może również wystąpić nieposłuszeństwo jednostek.
                        Więcej szczegółów znajduje się w podstawowych zasadach ww. gier z serii.
                  \item \textbf{pozostałe gry} --- Konfederacja zdobywa inicjatywę.
                \end{itemize}
          \item[B] \textbf{Segment aktywacji [4.3]} \par
                Gracz który zdobył inicjatywę musi wykonać pojedynczą akcję z wybranym dowódcą lub jednostką.
                Pod koniec segmentu aktywacji gracze wracają do segmentu inicjatywy i kontynuują cykl akcji.
                Gracze kontynuują wykonywanie akcji, dopóki oboje nie spasują w tym samym segmencie inicjatywy [4.4].
                
        \end{itemize}
  \item[4] \textbf{Faza Reorganizacji}\par
        Uprawnione do tego jednostki mogą wykonać umocnienia polowe, budować mosty, naprawiać mosty i przeprawy promowe
        oraz odpoczywać w celu redukcji poziomu zmęczenia, dezorganizacji, wyczerpania i demoralizacji.
  \item[5] \textbf{Koniec tury}\par Znacznik tury jest przesuwany na kolejne pole. Tura dobiega końca i rozpoczyna się kolejna tura gry.
        \textbf{Wyjątki:} Okazjonalnie szczegółowe zasady scenariusza mogą wymagać wykonania kolejnych mechanik gry.
\end{itemize}